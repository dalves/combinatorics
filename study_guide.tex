\documentclass[12pt]{article}

\usepackage{array}
\usepackage{adjustbox}
\usepackage{amsfonts}
\usepackage{amsmath}
\usepackage{amsthm}
\usepackage{amssymb}
\usepackage{dirtytalk}
\usepackage[a4paper, total={7in, 9in}]{geometry}
\usepackage{forest}
\usepackage{geometry}
\usepackage{multirow}
\usepackage{skak}
\usepackage{tikz}
\usepackage{titling}
\usepackage{wrapfig}
\usepackage{xcolor}

\def\multichoose#1#2{\ensuremath{\left(\kern-.3em\left(\genfrac{}{}{0pt}{}{#1}{#2}\right)\kern-.3em\right)}}

\newcommand*{\Scale}[2][4]{\scalebox{#1}{$#2$}}
\newcommand*{\SetHeight}[2]{\resizebox{!}{#1}{$#2$}}

\begin{document}
\pagenumbering{gobble}

\title{Math 142 Study Guide}
\date{2016-10-06}
\author{David Alves}

\begin{center}
\large \thetitle \\
by \theauthor \\
\end{center}
\section*{Notation \& Basics}

\newcommand{\term}[1]{\textbf{#1}&}
\def\arraystretch{2}
\begin{tabular}{ r p{24em} }
\term{set} A collection of elements where the order doesn't matter and every element is distinct (You can't have two elements that are equal to each other)\\
\term{list or $k-$tuple} A collection of elements $(a_1, a_2, \ldots, a_k)$ where the order matters and duplicates are allowed. For example $(1, 2)$ is different from $(2,1)$ and $(1,1)$ is a valid list.\\
\term{$a \in A$} $a$ is an element of the set $A$\\
\term{$a \notin A$} $a$ is \textbf{not} an element of the set $A$\\
\term{$|S|$} The number of elements in a set $S$ (\emph{cardinality} of $S$)\\
\term{$A \cap B$} The set of elements that are in both $A$ and $B$\\
\term{$A \cup B$} The set of elements that are either in $A$, in $B$, or in both\\
\term{$A \subset B$} The set $A$ contains only elements that are in $B$\\
\term{$A \times B$} The \emph{cartesian product} of sets $A$ and $B$, namely the set of every possible list $(a, b)$ where $a \in A$ and $b \in B$. Example: $A=\{1,2\}, B=\{3,4\},A \times B = \{(1, 3), (1, 4), (2, 3), (2, 4)\}$\\
\term{$\binom{n}{k}$} Number of subsets of size $k$ in a set of $n$ items, $\frac{n!}{k!(n-k)!}$\\
\term{domain} The set of values that can go into a function. By definition a function can be applied to every element of the domain\\
\term{codomain} That set of values that can come out of a function\\
\term{injective} Every element in the codomain is hit at \textbf{most} once\\
\term{surjective} Every element in the codomain is hit at \textbf{least} once\\
\term{bijective} Every element of the codomain is hit \textbf{exactly} once (both injective and surjective)\\
\term{$f : A \rightarrow B$} A function $f$ with domain $A$ and codomain $B$\\
\term{$[n]$} The set of integers $\{1, 2, \ldots, n-1, n\}$\\
\end{tabular}
\section*{Twelvefold Way with Balls and Bins}
\newcommand{\twelvecell}[2]{
    \begin{center}
        \minsizebox*{!}{.6cm}{$\displaystyle #1$}
    \end{center}
    #2
}
\newcommand{\twelverow}[1]{
    \centering #1
}

\newcommand{\finishrow}[0]{
    \medskip\\
    \hline
}

\def\arraystretch{1.2}
\lapbox[\width]{-.8cm}{
\begin{tabular}{ m{1.2cm}m{5.2cm}m{5.2cm}m{5.2cm}c }
     & \large\centering General           & \large\centering Injective                  & \large\centering Surjective&\\
     & \small\centering No restrictions & \small\centering At \textbf{most} one ball per bin & \small\centering At \textbf{least} one ball per bin&\\
    \hline
    
    
    \twelverow{$a$ balls\\(dist.)\bigskip\\$b$ bins\\ (dist.)} & 
    \twelvecell{b^a}{There are $b$ choices for each ball, so by the product principle there are $b^a$ total configurations.}& 
    \twelvecell{\prod_{i=0}^{a-1}b-i}{There are $b$ choices for the first ball, $b~-~1$ choices for the second ball (since it can't go in the same one), $b-2$ choices for the third, etc. ($\prod$ is like $\sum$ except you multiply the terms instead of adding them)} &
    \twelvecell{S(a,b)b!}{There are $S(a,b)$ ways to put distinguishable balls into indistinguishable bins by definition, and then $b!$ ways to label the bins after you have placed the balls.}
    \finishrow
    
    \twelverow{$a$ balls\\(indist.)\bigskip\\$b$ bins\\ (dist.)} & 
    \twelvecell{\multichoose{b}{a}=\binom{a+b-1}{a}}{Use $\star$ for a ball and $\vert$ for a divider between bins. There are $b-1$ dividers needed so you have a sequence of $a+b-1$ elements of which $a$ are balls, giving $\binom{a+b-1}{a}$. For example, $(0, 0, 3, 1)$ is $\vert\vert\star\star\star\vert\star$}&
    \twelvecell{\binom{b}{a}}{Number of ways to pick a subset of size $a$ from a set of size $b$. (We are picking $a$ of the bins to contain a single ball while the rest remain empty).}&
    \twelvecell{\binom{a-1}{a-b}}{Use the same stars-and-bars encoding as for general, except since each bin must have one, we can it out of the encoding. For example, $(3,1,2)$ is encoded as $\star\star\vert\vert\star$}
    \finishrow 
 
    \twelverow{$a$ balls\\(dist.)\bigskip\\$b$ bins\\ (indist.)} & 
    \twelvecell{\sum_{i=0}^{b} S(a,i)}{$S(a,i)$ is the number of ways to put $a$ balls into $i$ bins such that all bins contain at least one ball. Sum over all values of $i$ to get the total number of ways to fill $b$ bins without the \say{all bins used} requirement}&
    \twelvecell{
        \begin{aligned}    
            b\geq 1: 1\\
            b<a: 0
        \end{aligned}
    }{One way if there are enough bins, otherwise zero}&
    \twelvecell{S(a,b)}{$S(a,b)$ is defined to be the number of ways to place $a$ distinguishable balls into $b$ indistinguishable bins. Note that this is also the number of ways to partition a set with $a$ elements into $b$ groups.}
    \finishrow  
 
    \twelverow{$a$ balls\\(indist.)\bigskip\\$b$ bins\\ (indist.)} &  
    \twelvecell{\sum_{i=0}^{b} P(a,i)}{$P(a,i)$ is the number of ways to put $a$ balls into $i$ bins such that all bins contain at least one ball. We then sum over all values of $i$ from 0 to $b$ to get the total number of ways to fill $b$ bins without the \say{all bins used} requirement}&
    \twelvecell{
        \begin{aligned}    
            b\geq 1: 1\\
            b<a: 0
        \end{aligned}
    }{One way if there are enough bins, otherwise zero}&
    \twelvecell{P(a,b)}{This is defined to be the number of ways to place $a$ indistinguishable balls into $b$ indistinguishable bins. Note that this is also the number of ways to partition an integer $a$ into the sum of $b$ integers and the number of Young Diagrams with $a$ boxes and $b$ rows.}
\end{tabular}
}
\pagebreak
\section*{Equivalence Relations \& Partitions}
Coming soon...
\section*{Binomial Theorem}
Coming soon...
\section*{Principle of Inclusion-Exclusion}
Coming soon...
\section*{Catalan Numbers}
Coming soon...
%\[xC^2(x)-C(x) + 1 = 0\]
\section*{Generating Functions}
Coming soon...

\end{document}
