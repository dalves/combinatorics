\documentclass[12pt]{article}
\author{David Alves}

\usepackage{amsfonts}
\usepackage{amsmath}
\usepackage{amsthm}
\usepackage{dirtytalk}
\usepackage[a4paper, total={6.5in, 8.75in}]{geometry}
\usepackage{forest}
\usepackage{skak}
\usepackage{tikz}
\usepackage{titling}
\usepackage{wrapfig}
\usepackage{xcolor}

\def\multichoose#1#2{\ensuremath{\left(\kern-.3em\left(\genfrac{}{}{0pt}{}{#1}{#2}\right)\kern-.3em\right)}}

\title{Math 142 Problem Set 1}
\author{David Alves}
\date{2016-08-30}

\begin{document}
\pagenumbering{gobble}

\begin{center}
\large \thetitle \\
\theauthor \\
\thedate
\end{center}

\subsection*{Sources}

    \begin{itemize}
    \item http://tex.stackexchange.com and https://www.sharelatex.com for help with \LaTeX
    \end{itemize}

\section*{Subset Counting}

\subsection*{Problem Statement}
Prove that there are $2^{17}$ subsets of $[17]$
\subsection*{Solution}
Let $f(s)$ denote the set of all subsets of a set $s$. We can show that $|f([17])| = 2^{17}$ using induction by demonstrating two things:

\begin{enumerate}
\item $|f([0])| = 2^0$ (Base case)
\item If $|f([k])| = 2^k$, then $|f([k+1])| = 2^{k+1}$
\end{enumerate}

\subsubsection*{Base Case}
$[0]$ is the empty set, which contains no elements. The empty set is a subset of itself, but has no other subsets. Therefore $|f([0])| = 2^0$.
\subsubsection*{Inductive Step}
For each subset $s$ of $[k]$ there exist exactly two subsets of $[k+1]$: the original subset, and a subset consisting of the original subset plus the element not in $[k]$. Therefore $[k+1]$ contains exactly twice as many subsets as $[k]$, so $|f([k+1])| = 2|f([k])|$. Thus if $|f([k])| = 2^k$, then $|f([k+1])| = 2^{k+1}$ 

\end{document}
