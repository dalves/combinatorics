\documentclass[12pt]{article}
\author{David Alves}

\usepackage{amsmath}
\usepackage{amsthm}
\usepackage{dirtytalk}
\usepackage{forest}
\usepackage{skak}
\usepackage{tikz}
\usepackage{titling}

\title{Math 142 Problem Set 3}
\author{David Alves}
\date{2016-09-13}

\begin{document}
\pagenumbering{gobble}

\begin{center}
\Large \thetitle \\
\large \theauthor \\
\thedate
\end{center}

\subsection*{Sources}

    \begin{itemize}
    \item Notes from lecture and my own memory from Math 42 for solving the problems
    \item http://tex.stackexchange.com and https://www.sharelatex.com for help with \LaTeX
    \item Wikipedia articles on equivalence relations.
    \end{itemize}

\section{Equivalence Relations and Partitions}

\subsection*{Problem Statement}
Show that there is a bijection between the set of (equivalence relations on a set $A$) and the set of (partitions of $A$). 
\subsection*{Solution}

%{a b | c} <=> {(a,a), (b,b), (a,b), (b,a), (c,c)}

Let $P$ be a partition of $A$ and let $p_1, p_2, \ldots, p_n (n\geq 1)$ refer to the subsets in that partition. By the definition of a partition, each element of $A$ is contained in exactly one subset of $P$. Let $g(e) \in p_1, p_2, \ldots, p_n$ denote the subset that element $e$ is in.

Let $f(P)$ be the set of ordered pairs $\{(a, b) : g(a) = g(b)\}$. Note that $a$ and $b$ can be the same element in this definition. $f(P)$ is an equivalence relation, and $f$ is a bijection because it is both an injection and a surjection. Therefore there is a bijection between the set of (equivalence relations on a set $A$) and the set of (partitions of $A$).

\subsubsection*{Proof that $f(P)$ is an Equivalence Relation}
\begin{proof}
$f(P)$ is relation on $A\times A$ because it is a subset of $A\times A$ and satisfies the three requirements of an equivalence relation as follows:
\begin{itemize}
\item Reflexive property: $(e, e) \in f(P)$ for all $e \in A$. True because $g(e) = g(e)$.    
\item Symmetric property: If $(a, b)$ is in $f(P)$, $(b, a)$ must be in $f(P)$ for all $a, b \in A$. True because if $g(a) = g(b)$, then $g(b) = g(a)$.
\item Transitive property: If $(a,b)$ and $(b,c)$ are in $f(P)$, then $(a,c)$ is in  $f(P)$. True because if $g(a) = g(b)$ and $g(b) = g(c)$, then $g(a)$ = $g(c)$.
\end{itemize}

\end{proof}
\subsubsection*{Proof that $f$ is an Injection}
\begin{proof}
Suppose that $f$ is not an injection. Then there must be some two partitions $a,b$ of $A$ such that $f(a) = f(b)$ and $a \neq b$. Since $a \neq b$, there must be some two elements $e_1, e_2$ such that they are in the same group in one partition and in different groups in the other partition. Without loss of generality, suppose $e_1, e_2$ are in the same group in $a$ and different groups in $b$. Then $(a,b) \in f(a)$ and $(a,b) \notin f(b)$, which contradicts our assumption. Therefore our assumption is incorrect and $f$ must be an injection.
\end{proof}
\subsubsection*{Proof that $f$ is a Surjection}
\begin{proof}
Suppose that $f$ is not a surjection. Then there must be some equivalence relation $R$ such that this is no corresponding partition $p$ where $f(p) = R$. Let two members $a, b \in A$ be part of the same \emph{equivalence class} if $(a, b) \in R$. Select an element $e_1 \in A$. Let $e_1, \ldots, e_k \in R$ be the $k$ members of $e_1$'s \emph{equivalence class}. We know that $(e_1, e_1) \in R$ because $R$ satisfies the reflexive property. $(e_1, e_i), (e_1, e_j) \in R: i,j \leq k$ by definition, therefore $(e_i, e_1) \in R: i \leq k$ due to the symmetric property and $(e_i, e_j) \in R: i,j \leq k$ due to the transitive property. Therefore every edge between elements in $e_1, \ldots, e_k$ exists. 

There cannot be any edge $(f, e_i)$ or $(e_i, f) \in R$ such that $f \notin e_1, e_2, \ldots, e_k$ because $(e_1, f)$ would exist by the transitive property, therefore $f$ would be a member of $e_1$'s \emph{equivalence class}.

We can therefore construct a partition $p$ of $A$ such that $e_1, e_2, \ldots, e_k$ are in the same group. We know that all edges $(e_i, e_j): 1 \leq i, j \leq k$ exist in $f(p)$ and no edge $(f, e_i)$ or $(e_i, f)$ exists by the definition of $f$. Since we chose $e_1$ arbitrarily the same logic applies for all other groups. Therefore there exists a partition $p$ such that $f(p) = R$, which contradicts our assumption that $f$ is not a surjection. Therefore our assumption is false and $f$ is a surjection.
\end{proof}


\section{Amusement Park Counting}

There are $n$ couples. For each couple, there are two ways to select who sits on the left, giving $2^n$ choices by the product principle. For a roller coaster, there are $n!$ ways to order the couples in the cars, giving a total of $n!2^n$ arrangements. For a Ferris wheel, each arrangement of people is being overcounted $n$ times, one for each position that the Ferris wheel can be rotated into. Therefore we need to divide by $n$, giving a total of $(n-1)!2^n$ arrangements for a Ferris wheel.


\section{Binomial Coefficients}

\[
\sum_{\substack{k=0\\ \text{k odd}}}^{n}\binom{n}{k} = 
\sum_{\substack{k=0\\ \text{k even}}}^{n}\binom{n}{k} = 
\sum_{k=0}^{n-1}\binom{n}{k}
\]
\\
Below is an illustration to show why. $\sum_{k=0}^{n}\binom{n}{k}$ for alternate elements (either $k$ even or $k$ odd) is equal to the sum of the $\binom{n-1}{k}$ for all elements since each element on the $n-1$ row contributes to exactly one element on the following row (rather than both). 

\begin{center}
\begin{tikzpicture}[
    scale=1,
    tcancel/.append style={draw=#1, cross out, inner sep=1pt}
]
\node at ( 0, 0) {1};
\node at (-1,-1) {1};
\node at ( 1,-1) {1};
\node at (-2,-2) {1};
\node at ( 0,-2) {2};
\node at ( 2,-2) {1};
\node at (-3,-3) {1};
\node at (-1,-3) {3};
\node at ( 1,-3) {3};
\node at ( 3,-3) {1};
\node at ( 0, -4.5) {$\vdots$};
\node at (-6, -6) {1};
\node at (-4.5, -6) {$\ldots$};
\node at (-3, -6) (up2l) {$\binom{n-1}{k-2}$};
\node at (-1, -6) (up1l) {$\binom{n-1}{k-1}$};
\node at ( 1, -6) (up1r) {$\binom{n-1}{k}$};
\node at ( 3, -6) (up2r) {$\binom{n-1}{k+1}$};
\node at ( 4.5, -6) {$\ldots$};
\node at ( 6, -6) {1};
\node at (-7, -7) {1};
\node at (-5.5, -7) {$\ldots$};
\node at (-4, -7) (main2l) {$\binom{n}{k-2}$};
\node [color=gray,tcancel=red]at (-2, -7) (main1l) {$\binom{n}{k-1}$};
\node at ( 0, -7) (maincc)  {$\binom{n}{k}$};
\node [color=gray,tcancel=red] at ( 2, -7) (main1r){$\binom{n}{k+1}$};
\node at ( 4, -7) (main2r){$\binom{n}{k+2}$};
\node at (5.5, -7) {$\ldots$};
\node at ( 7, -7) {1};
\draw [->, thick] (up2r)--(main2r);
\draw [->, thick] (up1r)--(maincc);
\draw [->, thick] (up1l)--(maincc);
\draw [->, thick] (up2l)--(main2l);
\end{tikzpicture}
\end{center}


\section{Card Counting}

% With n ranks and k suits, how many ways are there to make 3 of a kind with a set of 5 cards?
% What about two pair?
% Which one is more rare for a normal deck?
% Give a deck where the rarity is switched

With $n$ ranks and $k$ suits, there are $nk^2\binom{k}{3}\binom{n-1}{2}$ ways to make 3-of-a-kind and $(n-2)k\binom{n}{2}\binom{k}{2}^2$ ways to make two pair. 

\subsection*{Three-of-a-kind}

Three-of-a-kind consists of three cards with identical rank and two cards which do not have that rank and do not share a rank with each other. There are $\binom{k}{3}$ ways to select the suits for the three cards and $n$ different possible ranks. There are $\binom{n-1}{2}$ ways to choose the ranks for the other two cards and $k$ possible suits for each of them. Therefore the total number of unique three-of-a-kind hands is 
\[
nk^2\binom{k}{3}\binom{n-1}{2}
\]

\subsection*{Two Pair}
For two pair, there are $\binom{n}{2}$ ways to choose the ranks for the two pairs and $n-2$ possible ranks for the remaining card. There are $\binom{k}{2}$ ways to select the suits for each of the two pairs. Finally there are $k$ possible suits for the remaining card. Therefore the total number of unique two pair hands is
\[
(n-2)k\binom{n}{2}\binom{k}{2}^2
\]

\subsection*{Rarity}
Plugging in the numbers for a normal ($n=13, k=4$) deck gives 10816 possible three-of-a-kind hands and 123552 possible two pair hands, therefore three-of-a-kind is more rare.

The rarity is switched for a deck with $n=3$ and $k=10$ because there are 108,000 unique 3-of-a-kind hands and 60,750 unique two pair hands.


\section{Fixed Elements}

\subsection*{Involutions and Fixed Points}

% Prove that the number of non-fixed points in an involution is even
\begin{proof}
For each non-fixed element, we have $f(a) = a', a \neq a'$. Since $f$ is an involution, we also know that $f(a') = a$. We can remove $a$ and $a'$ from the domain and codomain of $f$ respectively to construct $f'$. Repeatedly applying the same process we will eventually construct a function consisting of only zero or more fixed elements. After this process is complete we will have removed two elements from the domain $k$ times, meaning $2k$ total non-fixed elements existed. Therefore the the number of non-fixed elements in $f$ is even.\end{proof}

\subsection*{Odd divisors}

Note: I'm assuming that the problem is asking for positive divisors although that was not explicitly stated.


Let $n$ be a positive integer and let $D$ be the set of integers that evenly divide $n$. $|D|$ is odd if and only if $n$ is a perfect square, that is if there is some positive integer $k$ such that $k^2 = n$.
\begin{proof}
For each element $e$ of $D$, there exists an element $e'$ of $D$ such that $e*e' = n$ since $D$ contains all integer divisors of $n$. For each $e \neq e'$ we have exactly one corresponding element, so the total size of such elements is even. If $n$ is a perfect square then $D$ also contains a single element $e$ such that $e^2=n$, and therefore $|D|$ is odd if and only if $n$ is a perfect square.
\end{proof}
\section{Time Spent}

I spent about ten hours on this assignment. Most of them were pretty straightforward but \#1 took a very long time to do rigorously. (Maybe there's some clever shortcut?)

\end{document}
